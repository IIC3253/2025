%!TEX root = ../main/main.tex

En esta pregunta usted deberá implementar el protocolo de firmas de
anillo visto en clases, el cual utiliza las firmas de Schnorr también
vistas en clases.

Recuerde que una firma de Schnorr se define de la siguiente
forma. Suponga que los siguiente objetos son públicos: un grupo $(G,
*)$, un elemento $g \in G$, un número primo $q$ tal que $|\langle
g\rangle| = q$ y una función de hash $h$. La clave secreta de un
usuario $A$ es un número $x_A \in \{1, \ldots, q -1\}$ y su clave
pública es el elemento del grupo $y_A = g^{x_A}$. Si el
usuario $A$ quiere generar una firma de Schnorr para un mensaje $m$,
entonces debe realizar los siguientes pasos:
\begin{enumerate}
\item Genera al azar $r \in \{1, \ldots, q-1\}$ y calcula $c = h(g^r \| m)$ considerando $g^r$ como un string.

\item Calcula $s = r + c \cdot x_A$ interpretando $c$ como un número natural.

\item Define $(c, s)$ como la firma de Schnorr de $m$.
\end{enumerate}
Un usuario $B$ puede verificar si $(c,s)$ es una firma del mensaje
$m$ hecha por el usuario $A$ chequeando que la siguiente condición se
cumpla:
\begin{align*}
c \ = \ h(g^s * y_A^{q-c} \| m).
\end{align*}
Utilizando este protocolo, mostraremos cómo se define una firma de anillo para un caso particular. Usted deberá generalizarlo. Suponga que tiene un grupo formado por los usuarios $1$, $2$ y $3$, donde la clave secreta del usuario $i$ es $x_i$ y la clave pública del usuario $i$ es $y_i = g^{x_i}$. Si el usuario $1$ quiere generar una firma de anillo de un mensaje $m$, entonces debe realizar los siguientes pasos:
\begin{enumerate}
\item Genera al azar $r_1 \in \{1, \ldots, q-1\}$ y calcula $c_2 = h(g^{r_1} \| m)$.

\item Genera al azar $s_2 \in \{1, \ldots, q-1\}$.

\item Calcula $g^{r_2} = g^{s_2} * y_2^{q - c_2}$ y $c_3 = h(g^{r_2} \| m)$. Note que en este paso el usuario 1 no calcula $r_2$, sólo calcula $g^{s_2} * y_2^{q - c_2}$ y sabe que esto es igual a un elemento del grupo de la forma $g^{r_2}$. Además, el usuario 1 utiliza $g^{r_2}$ para calcular $c_3$.

\item Genera al azar $s_3 \in \{1, \ldots, q-1\}$.

\item Calcula $g^{r_3} = g^{s_3} * y_3^{q - c_3}$ y $c_1 = h(g^{r_3} \| m)$.

\item Calcula $s_1 = r_1 + c_1 \cdot x_1$. Note que este paso lo puede realizar el usuario $1$ ya que conoce la clave secreta $x_1$.

\item Saca al azar $i\in\{1,2,3\}$ y define $(s_1, s_2, s_3, c_i, i)$ como la firma de anillo de $m$.
\end{enumerate} 
Supongamos que $i=2$. Si un usuario $B$ quiere verificar que $(s_1, s_2, s_3, c_2, 2)$ es una firma de anillo válida para el grupo
formado por los usuarios $1$, $2$ y $3$, entonces debe hacer lo siguiente:

\begin{enumerate}
	\item Calcula $c_3=h(g^{s_2} * y_2^{q - c_2} \| m)$
	\item Calcula $c_1=h(g^{s_3} * y_3^{q - c_3} \| m)$
	\item Revisa si $h(g^{s_1} * y_1^{q - c_1} \| m)$ igual al valor $c_2$ de la firma.
\end{enumerate}
Note que si esta última condición es cierta, entonces $B$ sabe
que la firma es válida, pero no sabe cuál de los usuarios firmó el
mensaje $m$ (ya que el valor de $i$ se sacó al azar).

Para responder esta pregunta, usted debe entregar el notebook \texttt{pregunta1.ipynb} habiendo completado exclusivamente los bloques marcados con \texttt{\#\#\#\#\# POR COMPLETAR \#\#\#\#\#}. Su notebook deberá correr de principio a fin. Además, se evaluará con un programa externo su implementación de las clases \texttt{Signer} y \texttt{Verifier}.

\medskip

\paragraph{Corrección.} Para revisar esta tarea se van a ejecutar tres conjuntos de pruebas, en los cuales consideramos el código que usted entregó en su tarea y el código en la solución de la tarea:
\begin{itemize}
\item El primer conjunto de pruebas se utiliza para determinar si la clase \texttt{RingSetup} de la solución de la tarea funciona bien con las clases \texttt{Signer} y \texttt{Verifier} entregadas en su tarea.

\item El segundo conjunto de pruebas se utiliza para ver si la clase \texttt{Signer} entregada en su tarea genera firmas válidas de acuerdo a la clase \texttt{Verifier} de la solución de la tarea.

\item El tercer conjunto de pruebas se utiliza para determinar si la clase \texttt{Verifier} entregada en su tarea verifica correctamente firmas generadas por la clase \texttt{Signer} de la solución de la tarea.
\end{itemize}
Cada conjunto de pruebas consiste de 4 casos distintos:
\begin{itemize}
\item Una firma correcta para el grupo $\mathbb{Z}_p^*$.
\item Una firma incorrecta para el grupo $\mathbb{Z}_p^*$.

\item \begin{sloppypar} Una firma correcta para el grupo definido por la curva elíptica P-256 (\url{https://neuromancer.sk/std/nist/P-256}). \end{sloppypar}
\item Una firma incorrecta para el grupo definido por la curva elíptica P-256.
\end{itemize}
Por cada test ejecutado correctamente obtiene 5 décimas para la nota final de la tarea.

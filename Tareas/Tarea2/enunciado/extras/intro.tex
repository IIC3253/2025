%!TEX root = ../main/main.tex

\section*{Instrucciones}

Cualquier duda sobre la tarea se deberá hacer en los \emph{issues} del \href{https://github.com/IIC3253/2025}{repositorio del curso}. Los issues son el canal de comunicación oficial para todas las tareas.

\paragraph{Configuración inicial.} 
Para esta tarea utilizaremos \textit{github classroom}. 
Para acceder a su repositorio privado debe ingresar al siguiente \href{https://classroom.github.com/a/ZjiQ8NjP}{link}, seleccionar su nombre y aceptar la invitación.
El repositorio se creará automaticamente una vez que haga esto y lo podrá encontrar junto a los \href{https://github.com/orgs/IIC3253/repositories}{repositorios del curso}.
Para la corrección se utilizará Python 3.11.

\paragraph{Entrega.} Al entregar esta tarea, su repositorio se deberá ver exactamente de la siguiente forma:

\bigskip

\dirtree{%
.1 \faGithub \ Repositorio.
.2 \faFolderOpenO \ \texttt{Tarea2}.
.3 \faFolderOpenO \ \texttt{Pregunta1}.
.4  \faFilePdfO \ \texttt{pregunta1.pdf}.
.3 \faFolderOpenO \ \texttt{Pregunta2}.
.4 \faFilePdfO \ \texttt{pregunta2\_a.pdf}.
.4 \faFileCodeO \ \texttt{pregunta2\_b.py}.
.2 \faFileTextO \ \texttt{.gitignore}.
.2 \faFileTextO \ \texttt{README.md}.
.2 \faFolderO \ .git.
}

\bigskip

Para cada problema cuya solución se deba entregar como un documento (en este caso la pregunta 1), deberá entregar un archivo \texttt{.pdf} que, o bien fue construido utilizando \LaTeX, o bien es el resultado de digitalizar un documento escrito a mano. En caso de optar por esta última opción, queda bajo su responsabilidad la legibilidad del documento. Respuestas que no puedan interpretar de forma razonable los ayudantes y profesores, ya sea por la caligrafía o la calidad de la digitalización, serán evaluadas con la nota mínima.

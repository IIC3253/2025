%!TEX root = ../main/main.tex

En esta pregunta estudiaremos un sistema criptográfico de clave pública que es definido a partir de un problema NP-completo. Para esto considere el problema SUBSET-SUM definido de la siguiente forma. La entrada de SUBSET-SUM es un conjunto $S$ de números naturales y un número natural $b$, y la pregunta responder es si existe $I \subseteq S$ tal que:
\begin{eqnarray*}
  \sum_{a \in I} a &=& b.
\end{eqnarray*}
Por ejemplo, si $S_1 = \{1, 2, 3, 4, 5\}$ y $b_1 = 7$, entonces tanto $\{2, 5\}$ como $\{1 ,2, 4\}$ son soluciones del problema. Por el contrario, si $S_2 = \{2, 4, 6, 8\}$ y $b_2 = 5$, entonces el problema no tiene solución. Es conocido que SUBSET-SUM es un problema NP-completo, por lo que se cree que no existe un algoritmo de tiempo polinomial que lo solucione.

En el sistema criptográfico basado en SUBSET-SUM, un usuario establece sus claves pública y secreta a través del siguiente protocolo, suponiendo que el espacio de mensaje es \mbox{${\cal M} = \{0,1\}^n$}.
\begin{enumerate}
\item[(1)] Se escoge un conjunto $S = \{w_1, \ldots, w_n\}$ de números naturales tal que $S$ es súper creciente, vale decir, para cada $k \in \{2, \ldots, n\}$ se tiene que:
  \begin{eqnarray*}
    \sum_{i = 1}^{k-1} w_i &<& w_k.
  \end{eqnarray*}
  Por ejemplo, $S_1 = \{1, 2, 4, 8\}$ es súper creciente, mientras que $S_2 = \{1, 2, 3, 4\}$ no es súper creciente.


\item[(2)] Se escogen números naturales $q, r$ tales que $\textit{MCD}(q, r) = 1$ y 
  \begin{eqnarray*}
    \sum_{i=1}^n w_i &<& q.
  \end{eqnarray*}

  \begin{sloppypar}
\item[(3)] Se construye el conjunto $T = \{t_1, \ldots, t_n\}$ de números naturales tales que \mbox{$t_i = (w_i \cdot r) \text{ mod } q$} para cada $i \in \{1, \ldots, n\}$.
  \end{sloppypar}
 
\item[(4)] Se define la clave secreta como $(S,q,r)$ y la clave pública como $T$.
\end{enumerate}
Las funciones de cifrado y descifrado son definidas de la siguiente forma, considerando que la clave pública de un usuario $A$ es $P_A = T$ y la clave secreta de $A$ es $S_A = (S,q,r)$. Suponga que $m = b_1 \cdots b_n$ es un mensaje, donde $b_i \in \{0,1\}$ para cada $i \in \{1, \ldots, n\}$. Entonces
\begin{eqnarray*}
  \textit{Enc}_{P_A}(m) &=& \sum_{i = 1}^n b_i \cdot t_i.
\end{eqnarray*}
Observe que el texto cifrado de un mensaje es un número natural.

Por otro lado, la función de descifrado $\textit{Dec}_{S_A}$ se define de la siguiente forma. Dado un texto cifrado $c \in \mathbb{N}$, sea $d = (s \cdot c) \text{ mod } q$, donde $s$ es un número natural tal que $r \cdot s \equiv 1 \mod q$ (sabemos que este número existe puesto que $\textit{MCD}(q,r) = 1$). Además, sean $s_1$, $\ldots$, $s_n$ tales que $s_i \in \{0,1\}$ para cada $i \in \{1, \ldots, n\}$ y:
\begin{align}
  \sum_{i=1}^n s_i \cdot w_i \ = \ d. \tag{\dag} \label{eq-fin}
\end{align}
Entonces $\textit{Dec}_{S_A}(c) = s_1 \cdots s_n$.

A continuación usted debe demostrar que el protocolo anterior es correcto y puede ser implementado en tiempo polinomial, para lo cual debe responder las siguientes preguntas.
\begin{itemize}
\item[(a)] Demuestre que, dado un conjunto $S$ de números naturales que es súper creciente y un número $b$, existe a lo más un conjunto $I \subseteq S$ tal que:
  \begin{eqnarray*}
    \sum_{a \in I} a &=& b.
  \end{eqnarray*}

\item[(b)] Construya un algoritmo polinomial tal que, dado un conjunto $S$ de números naturales que es súper creciente y un número $b$, verifica si existe $I \subseteq S$ tal que:
  \begin{eqnarray*}
    \sum_{a \in I} a &=& b.
  \end{eqnarray*}
  Este algoritmo debe además construir $I$ si tal conjunto existe. Note que de (a) y (b) se concluye que $\textit{Dec}_{S_A}$ está bien definido, si existe una solución para la ecuación \eqref{eq-fin}, y puede ser implementado en tiempo polinomial.

\item[(c)] Sea $T = \{t_1, ..., t_n\}$ definido como en el paso (3) del protocolo. Demuestre que $t_i \neq t_j$ para $i,j \in \{1, \ldots, n\}$ tal que $i \neq j$. Note que de este se deduce que $T$ contiene $n$ elementos distintos y $\textit{Enc}_{P_A}$ está bien definido.

\item[(d)] Demuestre que para todo mensaje $m \in \{0,1\}^n$, se tiene que $\textit{Dec}_{S_A}(\textit{Enc}_{P_A}(m)) = m$.
\end{itemize}


\medskip

\paragraph{Corrección.} La asignación de puntajes para cada pregunta de este problema es la siguiente.
\begin{itemize}
\item[(a)] El puntaje máximo en esta pregunta es 1.5 puntos. Se obtiene 0.7 puntos si se entrega una versión de la demostración con algunos errores menores, y se obtiene el puntaje completo si la demostración no tiene errores.

\item[(b)] El puntaje máximo en esta pregunta es 1.5 puntos. Se obtiene 0.5
puntos si se entrega un algoritmo que funciona en tiempo polinomial
pero no se explica por qué es correcto, se obtiene 1 punto si se
entrega un algoritmo que funciona en tiempo polinomial y se explica
por qué es correcto, pero no se demuestra que se cumple esto, y se
obtiene 1.5 puntos si se entrega un algoritmo que funciona en tiempo
polinomial y se demuestra que es correcto.


\item[(c)] El puntaje máximo en esta pregunta es 1.5 puntos. Se obtiene 0.7 puntos si se entrega una versión de la demostración con algunos errores menores, y se obtiene el puntaje completo si la demostración no tiene errores.

\item[(d)] El puntaje máximo en esta pregunta es 1.5 puntos. Se obtiene 0.7 puntos si se da una idea de por qué es cierto que $\textit{Dec}_{S_A}(\textit{Enc}_{P_A}(m)) = m$ pero no se entrega una demostración completa de esta propiedad, y se obtiene el puntaje completo si se demuestra correctamente que esta propiedad es cierta.

\end{itemize}


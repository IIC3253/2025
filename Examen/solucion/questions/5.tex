Al usar RSA como sistema criptográfico, tenemos el problema de que si
encriptamos el mismo mensaje dos veces, obtenemos el mismo texto
cifrado. Para evitar este problema, suponiendo que las llaves pública
y secreta son $P_A=(e, N)$ y $S_A=(d, N)$, respectivamente, se
encripta un mensaje $m \in \{0, \ldots, N-1\}$ seleccionando al azar un número
$r \in \{1, \ldots, N-1\}$ que sea primo relativo con $N$, y
definiendo $\textit{Enc}_{P_A}(m) = ((m \cdot r)^e \text{ mod } N,\
r) = (c,r)$. Explique cómo se puede desencriptar $(c,r)$ teniendo acceso a
la llave secreta $S_A$.

\paragraph{Solución.} Para desencriptar $(c,r)$ basta con calcular $(c^d\cdot r^{-1}) \text{ mod } N$, puesto que esto es equivalente a $(c^d\text{ mod } N\cdot r^{-1}) \text{ mod } N$ y sabemos que $c^d\text{ mod } N=m\cdot r \text{ mod }N$. Dado que $r$ es primo relativo con $N$, podemos obtener $r^{-1}$ usando el algoritmo extendido de Euclides.

Enuncie el Teorema de Lagrange, y explique cómo este teorema nos ayuda
a verificar que el orden de un grupo generado $\langle g\rangle$ es
efectivamente un número $q$, cuando $q$ es primo.

\paragraph{Solución.} El Teorema de Lagrange dice que dado un grupo finito $G$ y un subgrupo $H$ de $G$, el orden de $H$ divide al orden de $G$. Para verificar que el orden de $\langle g\rangle$ es efectivamente $q$ lo que hacemos es verificar que $g$ sea distinto de la identidad y que $g^q$ sea la identidad. Como $g^q$ es la identidad, el orden de $\langle g\rangle$ es menor o igual a $q$. Pero como $\langle g\rangle$ es un grupo, su orden tendría que ser un divisor de $q$. Como $q$ es primo y $2\leq |\langle g\rangle|\leq q$, tenemos que $|\langle g\rangle|=q$.

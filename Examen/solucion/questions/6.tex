Suponga que está en un curso con 150 alumnos donde se necesita realizar las siguientes tareas:
\begin{itemize}
  \item[T1] Permitir a los alumnos enviar comentarios sobre el curso de manera anónima.

  \item[T2] Votar de manera voluntaria y anónima sobre algún aspecto del curso (por ejemplo, el cambio del horario de ayudantía)
\end{itemize}    
Para realizar estas tareas los profesores quieren utilizar el protocolo de firmas de anillo visto en el curso e implementado en la tarea 4. Conteste las siguientes preguntas, justificando su respuesta.
\begin{enumerate}
\item ¿Es una buena idea usar firmas de anillo para la tarea T1? 

\item ¿Es una buena idea usar firmas de anillo para la tarea T2? 
\end{enumerate}

\paragraph{Solución.}
\begin{enumerate}
\item Sí, ya que las firmas de anillo permitirían asegurar que los mensajes fueron enviados por alguien del curso, pero sin saber quién. Hay que considerar, eso sí, que cada alumno podría enviar tantos mensajes anónimos como quiera.

\item No, puesto que en una votación esperamos que cada persona pueda votar a lo más una vez, cosa que no podemos garantizar con el protocolo de firmas de anillos visto en este curso. Dicho de otra forma, si recibimos 100 votos, no podemos estar seguros de que esos 100 votos sean de 100 personas distintas. De hecho, una sola persona podría haber emitido esos 100 votos.
\end{enumerate}
